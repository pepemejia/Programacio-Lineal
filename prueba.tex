\documentclass{article}
\usepackage[utf8]{inputenc}
\usepackage{amsmath}
\usepackage[spanish]{babel}
\title{Apuntes de Programación Lineal}
\author{Israel Mejia}

\begin{document}



\maketitle
\tableofcontents
\section{Introducción}


La forma estándar de un problema de programción lineal es:
Dados una matriz $A$ y vectores $b,c$, maximizar $c^Tx$ sujeto a
$Ax\leq b$.

Variables de Holgura.

Las variables de holgura permiten determiar los excedentes de las
restricciones que podrían ser empleados en otros fines sin que la
solción óptima se altere.
Las variables de holgura permiten convertir las desigualdades en
igualdades


\begin{tabular}{|c|c|c|}
  &A&B\\
  \hline
  Maquina 1&1&2\\
  \hline
  Maquina 2&1&1

\end{tabular}
\begin{equation*}
  \label{eq:1}
 A= \begin{pmatrix}
    0&1&2\\
    1&5&6
  \end{pmatrix}
  \begin{pmatrix}
    1&2\\
    3&4\\
    3&7
  \end{pmatrix}
\end{equation*}
\end{document}













